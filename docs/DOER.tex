\hypertarget{description}{%
\section{Description}\label{description}}

Traditional GPS routing technologies are ubiquitous in today's
car-centred society. However, radical problems such as the climate and
heath emergencies require radical changes to the ways we get around.
Humans can traverse a much wider range of terrain than cars, yet mapping
systems rely solely on pre-defined paths and roads. In rural
environments, paths can be sparse, leaving those without experience with
an area to stick to longer, more arduous, and potentially dangerous
road-routes. Modern advances such as OS Maps and Google Maps' Terrain
View offer some terrain-sensitive routing information, such as
difficulty ranking, yet are still locked to the same pre-defined paths
and roads.

This project aims to create a more generalizable form of routing
algorithm; one not restricted to any pre-defined paths; and one that can
take into consideration the needs and abilities of different users.

\hypertarget{objectives}{%
\section{Objectives}\label{objectives}}

\hypertarget{primary}{%
\subsection{Primary}\label{primary}}

\hypertarget{modular-routing-algorithm}{%
\subsubsection{Modular Routing
Algorithm}\label{modular-routing-algorithm}}

The core artifact this project aims to create is the terrain sensitive
routing algrithm. This will take as input two points: the start and end
points which will be given using their latitude and longitude. The
output will give the optimal route based on a ranking algorithm. The
ranking algorithm will consider each potential step from the start to
target, and determine how ``good'' the considered steps are, where
``goodness'' will be defined as the predicted speed. It will store the
overal route ``goodness'', and progress along the route with the highest
score. At it's most basic, the following data sources must be
considered:

\begin{itemize}
\tightlist
\item
  \textbf{elevation/gradient} - both forwards and perpendicular gradient
  impact the viability and speed of a given route
\item
  \textbf{terrain type} - water and land is the most basic form of
  terrain distinction, but additional data sources will allow
  distinction between rocky terrain, forests, etc.
\item
  \textbf{distance to target} - some metric representing distance to the
  target point will be an essential component in determining the next
  optimal step on the route.
\end{itemize}

An essential design feature of this algorithm will be the ability to
adapt to the differing needs and abilities of it's users. The impact of
the different data sources available to the ranking algorihtm has on the
overal route rank will have to be explicitly defined. Representing these
as function curves, and allowing these to be created and customized will
allow users to change the importance of certain data on the feasibility
of a certain route.

\hypertarget{route-visualisation}{%
\subsubsection{Route Visualisation}\label{route-visualisation}}

The routing algorithm itself will need to output the suggested route.
However, to aid in the use of the algorithm for actual planning and
support for users following the route, the ability to see the output
projected onto the terrain is a desireable feature.

\hypertarget{generated-route-evaluation}{%
\subsubsection{Generated Route
Evaluation}\label{generated-route-evaluation}}

Ultimately, the usefulness of the routing algorithm will be determined
by the quality of the routes generated. Successful evaluation of the
suggested routes will aid in the creation of the ranking algorithm
configuration pre-sets. To do this, routes humans have already taken
between the same two points should be used to compare to those the
algorithm suggests.

\hypertarget{secondary}{%
\subsection{Secondary}\label{secondary}}

\hypertarget{more-complex-data-sources}{%
\subsubsection{More Complex Data
Sources}\label{more-complex-data-sources}}

To increase the usefulness of the algorithm, incorporating additional
data sources to the ranking algorithm will be useful as it allows more
realistic modelling of route planning. These data sources may include:

\begin{itemize}
\tightlist
\item
  \textbf{current weather conditions}
\item
  \textbf{historical weather conditions}
\item
  \textbf{water sources} - For long journeys, ensuring water sources are
  available at regular intervals is important.
\item
  \textbf{surrounding terrain gradient} - For many, traversing along a
  thin flat ridge with a sharp gradient adjacent is much slower than
  traversing along a large flat area
\end{itemize}

\hypertarget{ranking-configuration-presets}{%
\subsubsection{Ranking Configuration
Presets}\label{ranking-configuration-presets}}

In aid of allowing different mobility users to specify the terrain and
features impacting their routing decisions, creating presets for these
users would allow for less-experienced users to find utility in this
tool.

\hypertarget{tertiary}{%
\subsection{Tertiary}\label{tertiary}}

\hypertarget{interactive-configuration-interface}{%
\subsubsection{Interactive Configuration
Interface}\label{interactive-configuration-interface}}

To allow less technologically-experienced users to create their own
custom configurations, an interactive interface, allowing connections to
be made between data, and the definition of custom relationship curves
determining the data and it's impact on the speed of the potential route
will be useful.

\hypertarget{live-gps-tracking}{%
\subsubsection{Live GPS Tracking}\label{live-gps-tracking}}

Modern route finding tools like Google Maps allow live route tracking
which is useful as it helps those following the route to verify their
location along the route, and prevent getting lost. This objective may
be met by simply allowing the exporting of generated routes to a format
that can be imported by these live routing tools.

\hypertarget{animated-route-generation}{%
\subsubsection{Animated Route
Generation}\label{animated-route-generation}}

A technically interesting, but less useful feature to include is to
generate an animation (e.g.~GIF) which shows the routes considered and
decisions made by the ranking algorithm. This may give insights into the
thinking of the algorithm and even aid in determining further
improvements to the algorithm.

\hypertarget{ethics}{%
\section{Ethics}\label{ethics}}

Below this document contains the draft version of the full ethics
application, as I wish to use Strava Metro's API to evaluate the routes
generated by the algorithm against the most popular routes between the
same points.

\hypertarget{resources}{%
\section{Resources}\label{resources}}

No resources beyond standard lab provision required.

\hypertarget{risks}{%
\section{Risks}\label{risks}}

To evaluate the routes generated by the routing algorithm, I have
requested access to Strava Metro's API which offers a ``most popular
route'' between two points. This application is pending. To mitigate the
risk of this application failing or not being approved in time, if this
situation occurs, I have planned to go to popular off-road walking
destinations, and tracking my own routes to use to compare to the
generated routes.
