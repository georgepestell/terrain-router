% Options for packages loaded elsewhere
\PassOptionsToPackage{unicode}{hyperref}
\PassOptionsToPackage{hyphens}{url}
%
\documentclass[
]{article}
\usepackage{amsmath,amssymb}
\usepackage{lmodern}
\usepackage{iftex}
\ifPDFTeX
  \usepackage[T1]{fontenc}
  \usepackage[utf8]{inputenc}
  \usepackage{textcomp} % provide euro and other symbols
\else % if luatex or xetex
  \usepackage{unicode-math}
  \defaultfontfeatures{Scale=MatchLowercase}
  \defaultfontfeatures[\rmfamily]{Ligatures=TeX,Scale=1}
\fi
% Use upquote if available, for straight quotes in verbatim environments
\IfFileExists{upquote.sty}{\usepackage{upquote}}{}
\IfFileExists{microtype.sty}{% use microtype if available
  \usepackage[]{microtype}
  \UseMicrotypeSet[protrusion]{basicmath} % disable protrusion for tt fonts
}{}
\makeatletter
\@ifundefined{KOMAClassName}{% if non-KOMA class
  \IfFileExists{parskip.sty}{%
    \usepackage{parskip}
  }{% else
    \setlength{\parindent}{0pt}
    \setlength{\parskip}{6pt plus 2pt minus 1pt}}
}{% if KOMA class
  \KOMAoptions{parskip=half}}
\makeatother
\usepackage{xcolor}
\IfFileExists{xurl.sty}{\usepackage{xurl}}{} % add URL line breaks if available
\IfFileExists{bookmark.sty}{\usepackage{bookmark}}{\usepackage{hyperref}}
\hypersetup{
  pdfauthor={George Pestell (200007413)},
  hidelinks,
  pdfcreator={LaTeX via pandoc}}
\urlstyle{same} % disable monospaced font for URLs
\setlength{\emergencystretch}{3em} % prevent overfull lines
\providecommand{\tightlist}{%
  \setlength{\itemsep}{0pt}\setlength{\parskip}{0pt}}
\setcounter{secnumdepth}{-\maxdimen} % remove section numbering
\ifLuaTeX
  \usepackage{selnolig}  % disable illegal ligatures
\fi

\title{Terrain Sensitive Routing

Context Survey}
\author{George Pestell (200007413)}
\date{2024-10-05}

\begin{document}
\maketitle

\hypertarget{route-planners}{%
\section{Route Planners}\label{route-planners}}

Modern routing planner applications are used ubiquitously. The vast
majority of these focus on car-travel on roads, or limit pedestrian
users to pre-defined paths and pavements. The vast majority rank routes
based primarily on the shortest path between two points using the
network of paths and roads. Few effectively consider the terrain and
topography when ranking routes.

There are examples of routing planners which provide elevation data.
Google Maps' Terrain View has the ability to consider elevation data
along the path network. The Ordinance Survey's OSMaps gives users the
ability to see terrain difficulty rankings for given pre-defined routes.

\hypertarget{dems}{%
\section{DEMs}\label{dems}}

A key dataset a terrain-sensitive route planner must consider is
elevation data, as steep inclines are much more difficult to traverse
than flat planes. Digital Elevation Maps (DEMs) are raster data
containing a grid of elevation points across a terrain. Regular DEMs are
common, as the grid points are evenly spaced and so the X and Y
positions of each point are easily derived by calculating
\texttt{x\ =\ column\ *\ cell\_size,\ y\ =\ row\ *\ cell\_size} and

\hypertarget{mesh-derivation}{%
\section{Mesh Derivation}\label{mesh-derivation}}

As DEM data is given as a raster grid of elevation points, there is a
need to generate a traversable mesh representing that gradient data.
There are various ways to derive a mesh from a DEM, each with their own
benefits and drawbacks. The most popular method used in GIS applications
is the Triangular Irregular Network (TIN). TINs use a triangulation
algorithm to create a mesh where each face is a triangle. Delaunay
triangulation is a popular triangulation algorithm for this, as it aims
to limit very thin and long triangles, in favour of more equilateral,
evenly sized triangles. Another method for mesh derivation is the
square-based mesh-derivation. This creates a mesh consisting of
square/rectangle faces. This approach is particularly suited to regular
grid DEMs, because the data is equally spaced, and so the resulting mesh
consists of regularly sized square faces. Alternative mesh shapes can
also be used for mesh derivation, UBER's H2 positioning library shows
the potential to create a mesh from regular hexagons. Like with regular
square meshes, this has the benefit that each face is the same size, but
has the added benefit that the distance between hexagons is uniform in
all directions, unlike square meshes which have a different distance for
adjacent and diagonal faces, and unlike TINs which can have three
different distances: one for adjacent faces, and two different distances
for the diagonal faces.

\hypertarget{triangulation-algorithms}{%
\subsection{Triangulation Algorithms}\label{triangulation-algorithms}}

\hypertarget{delaunay-triangulation}{%
\subsection{Delaunay Triangulation}\label{delaunay-triangulation}}

Delaunay triangulation is the process of taking a set of 2D points and
forming a triangular mesh, given the constraints that the centre point
of each triangle should

\hypertarget{constrained-delaunay-triangulation}{%
\subsubsection{Constrained Delaunay
Triangulation}\label{constrained-delaunay-triangulation}}

\hypertarget{d-delaunay-triangulation}{%
\subsubsection{2.5-D Delaunay
triangulation}\label{d-delaunay-triangulation}}

\hypertarget{pre-existing-libraries}{%
\section{Pre-Existing Libraries}\label{pre-existing-libraries}}

A substantial amount of work has been put into creating libraries which
extract the complexities of creating an efficient Delaunay
implementation.

\begin{itemize}
\tightlist
\item
  \textbf{CGAL} - This suite of tools is open-source, and has
  simplification and smoothing tools also built-in.
\item
  \textbf{tin-terrain} - This open-source tool has been specifically
  created for the generation of TIN-terrains
\item
  \textbf{Fade25D} - This set of tools is especially tailored
\end{itemize}

\hypertarget{gis-software}{%
\section{GIS Software}\label{gis-software}}

The amount of data that can be represented on maps is immense. GIS is a
category of technologies which attempts to organize this data, and allow
for interesting data analysis. Data is categorized in layers
representing data on the map such as ``roads'', ``buildings'',
``rivers'', but there may also be more complex sub-categories.
Additionally, multiple data types (e.g.~buildings and rivers) may be
present in the same layer if required. These categories allow for
efficient filtering for specific applications.

Applications for GIS Software - TODO: EXPLAIN GIS SOFTWARE APPLICATIONS

\hypertarget{coordinate-systems}{%
\section{Coordinate Systems}\label{coordinate-systems}}

There are two primary categories of coordinate systems that can be used
to represent points on the earths surface.

\begin{itemize}
\tightlist
\item
  \textbf{Global Coordinate Systems}
\item
  \textbf{Projected Coordinate Systems}
\end{itemize}

Global coordinate systems represent x, y, and z positions of a
particular point on the earth viewed as a 3-dimensional sphere. An
example is the latitude, longitude coordinate system

\begin{itemize}
\tightlist
\item
  TODO: EXPLAIN GLOBAL COORDINATE SYSTEMS
\item
  TODO: EXPLAIN PROJECTED COORDINATE SYSTEMS
\end{itemize}

\hypertarget{dijkstra-a-algorithms}{%
\subsection{Dijkstra / A* Algorithms}\label{dijkstra-a-algorithms}}

A* is famous because it is often an efficient strategy that is
guaranteed to find the most optimal path given an admissible heuristic.
An admissible heuristic is one that never over-estimates the difficulty
of a given step. However, constructing such an admissible heuristic can
be challenging given a multifactorial routing algorithm. Dijkstra's
algorithm is a generalization of the A* algorithm, allowing no
admissible heuristic to be defined.

TODO: EXPLAIN A* TODO: EXPLAIN DIJKSTRA's ALGORITHM

\end{document}
