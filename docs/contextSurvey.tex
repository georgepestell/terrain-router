% Options for packages loaded elsewhere
\PassOptionsToPackage{unicode}{hyperref}
\PassOptionsToPackage{hyphens}{url}
%
\documentclass[
]{article}
\usepackage{amsmath,amssymb}
\usepackage{lmodern}
\usepackage{iftex}
\ifPDFTeX
  \usepackage[T1]{fontenc}
  \usepackage[utf8]{inputenc}
  \usepackage{textcomp} % provide euro and other symbols
\else % if luatex or xetex
  \usepackage{unicode-math}
  \defaultfontfeatures{Scale=MatchLowercase}
  \defaultfontfeatures[\rmfamily]{Ligatures=TeX,Scale=1}
\fi
% Use upquote if available, for straight quotes in verbatim environments
\IfFileExists{upquote.sty}{\usepackage{upquote}}{}
\IfFileExists{microtype.sty}{% use microtype if available
  \usepackage[]{microtype}
  \UseMicrotypeSet[protrusion]{basicmath} % disable protrusion for tt fonts
}{}
\makeatletter
\@ifundefined{KOMAClassName}{% if non-KOMA class
  \IfFileExists{parskip.sty}{%
    \usepackage{parskip}
  }{% else
    \setlength{\parindent}{0pt}
    \setlength{\parskip}{6pt plus 2pt minus 1pt}}
}{% if KOMA class
  \KOMAoptions{parskip=half}}
\makeatother
\usepackage{xcolor}
\IfFileExists{xurl.sty}{\usepackage{xurl}}{} % add URL line breaks if available
\IfFileExists{bookmark.sty}{\usepackage{bookmark}}{\usepackage{hyperref}}
\hypersetup{
  pdfauthor={200007413},
  hidelinks,
  pdfcreator={LaTeX via pandoc}}
\urlstyle{same} % disable monospaced font for URLs
\setlength{\emergencystretch}{3em} % prevent overfull lines
\providecommand{\tightlist}{%
  \setlength{\itemsep}{0pt}\setlength{\parskip}{0pt}}
\setcounter{secnumdepth}{-\maxdimen} % remove section numbering
\ifLuaTeX
  \usepackage{selnolig}  % disable illegal ligatures
\fi

\author{200007413}
\date{2024-10-05}

\begin{document}

This section contains an overview and evaluation of the pre-existing
works related to this project. It will serve as a foundation to the
design and implementation decisions.

\hypertarget{route-planners}{%
\section{Route Planners}\label{route-planners}}

Modern routing planner applications are used ubiquitously. The vast
majority of these focus on car-travel on roads, or limit pedestrian
users to pre-defined paths and pavements. The vast majority rank routes
based primarily on the shortest path between two points using the
network of paths and roads. Few effectively consider the terrain and
topography when ranking routes.

There are examples of routing planners which provide elevation data.
Google Maps' Terrain View has the ability to consider elevation data
along the path network. The Ordinance Survey's OSMaps gives users the
ability to see terrain difficulty rankings for given pre-defined routes.

\hypertarget{dems}{%
\section{DEMs}\label{dems}}

A key dataset a terrain-sensitive route planner must consider is
elevation data. Steep inclines are much more difficult to traverse
than\\
\# Delaunay Triangulation

\hypertarget{gis-software}{%
\section{GIS Software}\label{gis-software}}

The amount of data that can be represented on maps is immense. GIS is a
category of technologies which attempts to organize this data, and allow
for interesting data analysis. Data is categorized in layers
representing data on the map such as ``roads'', ``buildings'',
``rivers'', but there may also be more complex sub-categories.
Additionally, multiple data types (e.g.~buildings and rivers) may be
present in the same layer if required. These categories allow for
efficient filtering for specific applications.

Applications for GIS sog

\end{document}
